%% -*- coding: utf-8 -*-
\documentclass[12pt,a4paper]{scrartcl} 
\usepackage[utf8]{inputenc}
\usepackage[english,russian]{babel}
\usepackage{indentfirst}
\usepackage{misccorr}
\usepackage{graphicx}
\usepackage{amsmath}
\begin{document}
	\begin{titlepage}
		\begin{center}
			\large
			МИНИСТЕРСТВО НАУКИ И ВЫСШЕГО ОБРАЗОВАНИЯ РОССИЙСКОЙ ФЕДЕРАЦИИ
			
			Федеральное государственное бюджетное образовательное учреждение высшего образования
			
			\textbf{АДЫГЕЙСКИЙ ГОСУДАРСТВЕННЫЙ УНИВЕРСИТЕТ}
			\vspace{0.25cm}
			
			Инженерно-физический факультет
			
			Кафедра автоматизированных систем обработки информации и управления
			\vfill

			\vfill
			
			\textsc{Отчет по практике}\\[5mm]
			
			{\LARGE \textit{Найти определитель матрицы.}}
			\bigskip
			
			2 курс, группа 2ИВТ
		\end{center}
		\vfill
		
		\newlength{\ML}
		\settowidth{\ML}{«\underline{\hspace{0.7cm}}» \underline{\hspace{2cm}}}
		\hfill\begin{minipage}{0.5\textwidth}
			Выполнил:\\
			\underline{\hspace{\ML}} Н.\,А.~Свериденко\\
			«\underline{\hspace{0.7cm}}» \underline{\hspace{2cm}} 2021 г.
		\end{minipage}%
		\bigskip
		
		\hfill\begin{minipage}{0.5\textwidth}
			Руководитель:\\
			\underline{\hspace{\ML}} С.\,В.~Теплоухов\\
			«\underline{\hspace{0.7cm}}» \underline{\hspace{2cm}} 2021 г.
		\end{minipage}%
		\vfill
		
		\begin{center}
			Майкоп, 2021 г.
		\end{center}
	\end{titlepage}
	\section{Введение}
\label{sec:intro}

\begin{enumerate}
 \item Нужно написать программу, которая вычисляет определитель матрицы.
 \item Пример кода, решающего данную задачу
 \item Скриншот программы
\end{enumerate}

Пример приведен в пункте~\ref{sec:exp} в пункте~\pageref{sec:exp}.

\section{Ход работы}
\label{sec:exp}

\subsection{Код приложения}
\label{sec:exp:code}
\begin{verbatim}

% сурс
\end{verbatim}

\subsection{Пример задачи}
\label{sec:mathexample}

Нахождение определителя матрицы 3х3:
\begin{equation}\label{eq:solv}
 x_{1,2}=\frac{-b\pm\sqrt{b^2-4ac}}{2a}
\end{equation}

\section{Скриншот программы}
\label{sec:picexample}
\begin{figure}[h]
	\centering
	\includegraphics[width=0.4\textwidth]{parabola.jpg}
	\caption{Парабола}\label{fig:par}
\end{figure}

Пример параболы представлен на рис.~\ref{fig:par}.

\section{Пример библиографических ссылок}

Для изучения «внутренностей» \TeX{} необходимо 
изучить~\cite{Knuth-2003}, а для использования \LaTeX{} лучше
почитать~\cite{Lvovsky-2003, Voroncov-2005}.

\begin{thebibliography}{9}
\bibitem{Knuth-2003}Кнут Д.Э. Всё про \TeX. \newblock --- Москва: Изд. Вильямс, 2003 г. 550~с.
\bibitem{Lvovsky-2003}Львовский С.М. Набор и верстка в системе \LaTeX{}. \newblock --- 3-е издание, исправленное и дополненное, 2003 г.
\bibitem{Voroncov-2005}Воронцов К.В. \LaTeX{} в примерах. 2005 г.
\end{thebibliography}

\end{document}
